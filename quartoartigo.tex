\documentclass{article}
% Introdução ao LaTeX
% Seminário LaTeX -- o Livro
% Geraldo Xexéo
% Este arquivo tem a licença Creative Commons
% BY-NC-SA 2020

\usepackage[T1]{fontenc}
\usepackage[english,brazilian]{babel}
\usepackage[backend=biber,style=authoryear,natbib]{biblatex}
\addbibresource{references.bib}
\title{Meu Quarto Artigo}
\author{Geraldo Xexéo}
\begin{document}
\maketitle
\citet{biber:2012} não fala nada sobre isso. Mas \citep{bibera2012} também não. Você pode fazer uma equação in line $\sin(\theta)^\pi)$ ou citar
a equação \ref{eq:cita} que vai ficar destacada no texto.
\begin{equation}\label{eq:cita}
  y = \sum^{N}_{i=1} \cos(\alpha_i^\rho)+3\times \log(x)
\end{equation}
\printbibliography       
\end{document}